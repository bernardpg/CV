%%%%%%%%%%%%%%%%% PREAMBLE %%%%%%%%%%%%%%%%%%%%%%%%%%%%
%Change the font size of your document - 10pt, 12.1pt, etc.
\documentclass[letterpaper,11pt,oneside]{article}
\usepackage[utf8]{inputenc}
\usepackage{setspace}
\usepackage{hyperref}

\usepackage{graphicx}
\graphicspath{ {images/}} %upload your signature to this file
%Change the margins to fit your CV/resume content
\usepackage[left=1in, right=1in, bottom=1.25in, top=1.25in]{geometry}

%Skype information - include your Skype name for a link to add you on Skype
\newcommand*{\Skype}{\href{skype:scottpiliben?add}{scottpiliben}} 
\newcommand{\Absender}[1][\normalsize]{\Skype} 

%Changes the page numbers - {arabic}=arabic numerals, {gobble}=no page numbers, {roman}=Roman numerals
\pagenumbering{gobble}

%%%%%%%%%%%%%%%%% END OF PREAMBLE %%%%%%%%%%%%%%%%%%%%%

\begin{document}

%%%%%%%%%%%%%%%%% NAME OF APPLICANT %%%%%%%%%%%%%%%%%%%

\noindent  \LARGE{\textbf{Yen-Cheng Bernard Huang}}  \\
\vspace{-2ex}
\hline 
\normalsize

%%%%%%%%%%%%%%%%% CONTACT INFORMATION %%%%%%%%%%%%%%%%%
% Your email address, website, and Skype name are links to send email, open your website and add you on Skype. 

\begin{center}
\begin{tabular}
\href {Mail to : scottpiliben@gmail.com} \\
\hspace{1in} Skype: \Absender  \\
\hspace{1in} Phone: (+886)983-577-597 \\
\end{tabular}
\end{center}
\hline 
\vspace{1em}

%%%%%%%%%%%%%%%%% MAIN BODY %%%%%%%%%%%%%%%%%%%%%%%%%%%
% The main body is contained in a tabular environment. To move sections onto the next page, simply end the tabular environment and begin a new tabular environment.

\noindent \begin{tabular}{@{} l l}
 \Large{Education}    & \textbf{Most Recent University} \\
     & M.A, Physics, 2018. \\
     & Fields: Machine Learning, Pattern recognition, CNN \\
     & B.A., Physics, 2016. \\
     &\\
     & \textbf{National Sun Yat-sen University} \\
 \Large{Thesis}    & ``The analysis of single neuron branch point within Drosophila brain" \\
    & \parbox{5.0in}{This study could show some potential relationship}\\
    & \textbf{ within the
       branch-point number and its morphology.}\\
    & \textbf{Utilizing CNN verifies the image of the   neuron number. }\\
    & \\
 \Large{Proficient Skills  }    & \textbf{Data analysis tools: } \\
    & \textbf{EEG analyze : EEGLAB Curry8} \\
    & \textbf{Programming languages  : Matlab/Python/C} \\
 \Large{And Knowledge}  \\
    &\textbf{Selective Courses: Stochastics Simulation Methods}  \\
    &\textbf{and Its Application In SJTU}\\
    &\\
    &\textbf{Brain-Computer Interface : Principles and Practice}
&\\
\Large{Research Interests  } & \textbf{Neuroimaging (EEG,Graph Theory(Network))  } \\
& \textbf{Machine Learning(CNN...)} \\
\Large{Awards}\\
& \textbf{Great Performance in Neuroscience summer intern 2016}\\
& \textbf{Topic : Stochastic Resonance in Zebra Fish}\\
& \textbf{Supervisor: C. K. Chan}\\
& \textbf{(Institute of Physics in Academic Sinica) }\\
&\\
&\\
\Large{Reach and Work Experience}\\
& \textbf{ Neuroscience summer intern in Academia Sinica 2016}\\
 &\textbf{ Summer Exchange student in SJTU,Shanghai 2017}
\end{tabular}


\end{document}

